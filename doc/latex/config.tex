\chapter{Server config file format}
\label{config}

\section{General format}

	The config file consists of a series of BIND-{}style blocks. Each block
	consists of a series of values inside it which pertain to configuration
	settings that apply to the given block.

	Several values take lists of values and have defaults preset inside
	them. Prefix a keyword with a tilde (\textasciitilde{}) to override the
	default and disable it.

	A line may also be a .include directive, which is of the form
\begin{verbatim}.include "file"\end{verbatim}
	and causes \userliteral{file} to be read in at that point, before the
	rest of the current file is processed. Relative paths are first tried
	relative to PREFIX and then relative to ETCPATH (normally PREFIX/etc).

	Anything from a \# to the end of a line is a comment. Blank lines are
	ignored. C-{}style comments are also supported.


\section{Specific blocks and directives}
\label{configlines}

	Not all configuration blocks and directives are listed here, only the
	most common ones. More blocks and directives will be documented in
	later revisions of this manual.


\subsection{loadmodule directive}

\begin{verbatim}
loadmodule "text";\end{verbatim}

	Loads a module into the IRCd.


\subsection{serverinfo \{\} block}

\begin{verbatim}
serverinfo {
	name = "text";
	sid = "text";
	description = "text";
	network_name = "text";
	network_desc = "text";
	hub = boolean;
	vhost = "text";
	vhost6 = "text";
};\end{verbatim}

	The serverinfo \{\} block defines the core operational parameters of
	the IRC server.


{\sc serverinfo \{\} variables}
\nopagebreak

\noindent
\begin{description}
\item[{name}]
	The name of the IRC server that you are configuring. This must contain
	at least one dot. It is not necessarily equal to any DNS name. This
	must be unique on the IRC network.

\item[{sid}]
	A unique ID which describes the server.
	This consists of one digit and two characters which can be digits or
	letters.

\item[{description}]
	A user-{}defined field of text which describes the IRC server.
	This information is used in /links and /whois requests. Geographical
	location information could be a useful use of this field, but most
	administrators put a witty saying inside it instead.

\item[{network\_name}]
	The name of the IRC network that this server will be a member of.
	This is used in the welcome message and NETWORK= in 005.

\item[{network\_desc}]
	A description of the IRC network that this server will be a member of.
	This is currently unused.

\item[{hub}]
	A boolean which defines whether or not this IRC server will be serving
	as a hub, i.e. have multiple servers connected to it.

\item[{vhost}]
	An optional text field which defines an IP from which to connect
	outward to other IRC servers.

\item[{vhost6}]
	An optional text field which defines an IPv6 IP from which to connect
	outward to other IRC servers.
\end{description}


\subsection{admin \{\} block}

\begin{verbatim}
admin {
	name = "text";
	description = "text";
	email = "text";
};\end{verbatim}

	This block provides the information which is returned by the ADMIN
	command.


{\sc admin \{\} variables}
\nopagebreak

\noindent
\begin{description}
\item[{name}]
	The name of the administrator running this service.

\item[{description}]
	The description of the administrator's position in the network.

\item[{email}]
	A point of contact for the administrator, usually an e-{}mail address.
\end{description}


\subsection{class \{\} block}

\begin{verbatim}
class "name" {
	ping_time = duration;
	number_per_ident = number;
	number_per_ip = number;
	number_per_ip_global = number;
	cidr_ipv4_bitlen = number;
	cidr_ipv6_bitlen = number;
	number_per_cidr = number;
	max_number = number;
	sendq = size;
};\end{verbatim}

\begin{verbatim}
class "name" {
	ping_time = duration;
	connectfreq = duration;
	max_number = number;
	sendq = size;
};\end{verbatim}

	Class blocks define classes of connections for later use. The class
	name is used to connect them to	other blocks in the config file
	(auth\{\} and connect\{\}). They must be defined before they are used.

	Classes are used both for client and server connections, but most
	variables are different.


{\sc class \{\} variables: client classes}
\nopagebreak

\noindent
\begin{description}
\item[{ping\_time}]
	The amount of time between checking pings for clients, e.g.: 2 minutes

\item[{number\_per\_ident}]
	The amount of clients which may be connected from a single identd
	username on a per-{}IP basis, globally. Unidented clients all count as
	the same username.

\item[{number\_per\_ip}]
	The amount of clients which may be connected from a single IP address.

\item[{number\_per\_ip\_global}] The amount of clients which may be connected
	globally from a single IP address.

\item[{cidr\_ipv4\_bitlen}]
	The netblock length to use with CIDR-{}based client limiting for IPv4
	users in this class (between 0 and 32).

\item[{cidr\_ipv6\_bitlen}]
	The netblock length to use with CIDR-{}based client limiting for IPv6
	users in this class (between 0 and 128).

\item[{number\_per\_cidr}]
	The amount of clients which may be connected from a single netblock.

	If this needs to differ between IPv4 and IPv6, make different classes
	for IPv4 and IPv6 users.

\item[{max\_number}]
	The maximum amount of clients which may use this class at any given time.

\item[{sendq}]
	The maximum size of the queue of data to be sent to a client before it
	is dropped.
\end{description}

{\sc class \{\} variables: server classes}
\nopagebreak

\noindent
\begin{description}
\item[{ping\_time}] The amount of time between checking pings for servers, e.g.: 2 minutes
\item[{connectfreq}] The amount of time between autoconnects. This must at least be one minute, as autoconnects are evaluated with that granularity.
\item[{max\_number}] The amount of servers to autoconnect to in this class. More precisely, no autoconnects are done if the number of servers in this class is greater than or equal max\_number
\item[{sendq}] The maximum size of the queue of data to be sent to a server before it is dropped.
\end{description}

\subsection{auth \{\} block}

\begin{verbatim}
auth {
	user = "hostmask";
	password = "text";
	spoof = "text";
	flags = list;
	class = "text";
};\end{verbatim}

	auth \{\} blocks allow client connections to the server, and set
	various properties concerning those connections.

	Auth blocks are evaluated from top to bottom in priority, so put
	special blocks first.


{\sc auth \{\} variables}
\nopagebreak

\noindent
\begin{description}
\item[{user}]
	A hostmask (user@host) that the auth \{\} block applies to. It is
	matched against the hostname and IP address (using :: shortening for
	IPv6 and prepending a 0 if it starts with a colon) and can also use
	CIDR masks. You can have multiple user entries.

\item[{password}]
	An optional password to use for authenticating into this auth\{\}
	block. If the password is wrong the user will not be able to connect
	(will not fall back on another auth\{\} block).

\item[{spoof}]
	An optional fake hostname (or user@host) to apply to users
	authenticated to this auth\{\} block. In STATS i and TESTLINE, an
	equals sign (=) appears before the user@host and the spoof is shown.

\item[{flags}]
	A list of flags to apply to this auth\{\} block. They are listed below.
	Some of the flags appear as a special character, parenthesized in the
	list, before the user@host in STATS i and TESTLINE.

\item[{class}]
	A name of a class to put users matching this auth\{\} block into.
\end{description}

{\sc auth \{\} flags}
\nopagebreak

\noindent
\begin{description}
\item[{encrypted}]
	The password used has been encrypted.

\item[{spoof\_notice}]
	Causes the IRCd to send out a server notice when activating a spoof
	provided by this auth\{\} block.

\item[{exceed\_limit (>{})}]
	Users in this auth\{\} block can exceed class-{}wide limitations.

\item[{dnsbl\_exempt (\$)}]
	Users in this auth\{\} block are exempted from DNSblacklist checks.
	However, they will still be warned if they are listed.

\item[{kline\_exempt (\^{})}]
	Users in this auth\{\} block are exempted from DNS blacklists, k:lines
	and x:lines.

\item[{spambot\_exempt}]
	Users in this auth\{\} block are exempted from spambot checks.

\item[{shide\_exempt}]
	Users in this auth\{\} block are exempted from some serverhiding
	effects.

\item[{jupe\_exempt}]
	Users in this auth\{\} block do not trigger an alarm when joining
	juped channels.

\item[{resv\_exempt}]
	Users in this auth\{\} block may use reserved nicknames and channels. Note
	that the initial nickname may still not be reserved.

\item[{flood\_exempt (|)}]
	Users in this auth\{\} block may send arbitrary amounts of commands per
	time unit to the server. This does not exempt them from any other flood
	limits.	You should use this setting with caution.

\item[{no\_tilde (-{})}]
	Users in this auth\{\} block will not have a tilde added to their
	username if they do not run identd.

\item[{need\_ident (+)}]
	Users in this auth\{\} block must have identd, otherwise they will be
	rejected.

\item[{need\_ssl}]
	Users in this auth\{\} block must be connected via SSL/TLS, otherwise
	they will be rejected.

\item[{need\_sasl}]
	Users in this auth\{\} block must identify via SASL, otherwise they
	will be rejected.
\end{description}


\subsection{exempt \{\} block}

\begin{verbatim}
exempt {
	ip = "ip";
};\end{verbatim}

	An exempt block specifies IP addresses which are exempt from D:lines
	and throttling. Multiple addresses can be specified in one block.
	Clients coming from these addresses can still be K/G/X:lined or banned
	by a DNS blacklist unless they also have appropriate flags in their
	auth\{\} block.


{\sc exempt \{\} variables}
\nopagebreak

\noindent
\begin{description}
\item[{ip}]
	The IP address or CIDR range to exempt.
\end{description}

\subsection{privset \{\} block}

\begin{verbatim}
privset {
	extends = "name";
	privs = list;
};\end{verbatim}

	A privset (privilege set) block specifies a set of operator privileges.


{\sc privset \{\} variables}
\nopagebreak

\noindent
\begin{description}
\item[{extends}]
	An optional privset to inherit. The new privset will have all privileges that the given privset has.

\item[{privs}]
	Privileges to grant to this privset. These are described in the operator privileges section.
\end{description}


\subsection{operator \{\} block}

\begin{verbatim}
operator "name" {
	user = "hostmask";
	password = "text";
	rsa_public_key_file = "text";
	umodes = list;
	snomask = "text";
	flags = list;
};\end{verbatim}

	Operator blocks define who may use the OPER command to gain extended privileges.


{\sc operator \{\} variables}
\nopagebreak

\noindent
\begin{description}
\item[{user}]
	A hostmask that users trying to use this operator \{\} block must
	match. This is checked against the original host and IP	address; CIDR
	is also supported. So auth \{\} spoofs work in operator \{\} blocks;
	the real host behind them is not checked. Other kind of spoofs do not
	work in operator \{\} blocks; the real host behind them is checked.

\item[{password}]
	A password used with the OPER command to use this operator \{\} block.
	Passwords are encrypted by default, but may be unencrypted if
	\textasciitilde{}encrypted is present in the flags list.

\item[{rsa\_public\_key\_file}]
	An optional path to a RSA public key file associated with the
	operator \{\} block. This information is used by the CHALLENGE command,
	which is an alternative authentication scheme to the traditional OPER
	command.

\item[{umodes}]
	A list of usermodes to apply to successfully opered clients.

\item[{snomask}]
	An snomask to apply to successfully opered clients.

\item[{privset}]
	The privilege set granted to successfully opered clients. This must be
	defined before this operator\{\} block.

\item[{flags}] A list of flags to apply to this operator\{\} block. They are listed below.
\end{description}

{\sc operator \{\} flags}
\nopagebreak

\noindent
\begin{description}
\item[{encrypted}]
	The password used has been encrypted. This is enabled by default, use
	\textasciitilde{}encrypted to disable it.

\item[{need\_ssl}]
	Restricts use of this operator\{\} block to SSL/TLS connections only.
\end{description}


\subsection{connect \{\} block}

\begin{verbatim}
connect "name" {
	host = "text";
	send_password = "text";
	accept_password = "text";
	port = number;
	hub_mask = "mask";
	leaf_mask = "mask";
	class = "text";
	flags = list;
	aftype = protocol;
};\end{verbatim}

	Connect blocks define what servers may connect or be connected to.

{\sc connect \{\} variables}
\nopagebreak

\noindent
\begin{description}
\item[{host}]
	The hostname or IP to connect to. Furthermore, if a hostname is used,
	it must have an A or AAAA record (no CNAME) and it must be the primary
	hostname for inbound connections to work.

	IPv6 addresses must be in :: shortened form; addresses which then start
	with a colon must be prepended with a zero, for example 0::1.

\item[{send\_password}]
	The password to send to the other server.

\item[{accept\_password}]
	The password that should be accepted from the other server.

\item[{port}]
	The port on the other server to connect to.

\item[{hub\_mask}]
	An optional domain mask of servers allowed to be introduced by this
	link. Usually, "*" is fine. Multiple hub\_masks may be specified, and
	any of them may be introduced. Violation of hub\_mask and leaf\_mask
	restrictions will cause the local link to be closed.

\item[{leaf\_mask}]
	An optional domain mask of servers not allowed to be introduced by this
	link. Multiple leaf\_masks may be specified, and none of them may be
	introduced. leaf\_mask has priority over hub\_mask.

\item[{class}]
	The name of the class this server should be placed into.

\item[{flags}]
	A list of flags concerning the connect block. They are listed below.

\item[{aftype}]
	The protocol that should be used to connect with, either ipv4 or ipv6.
	This defaults to ipv4 unless host is a numeric IPv6 address.
\end{description}


{\sc connect \{\} flags}
\nopagebreak

\noindent
\begin{description}
\item[{encrypted}]
	The value for accept\_password has been encrypted.

\item[{autoconn}]
	The server should automatically try to connect to the server defined
	in this connect \{\} block if it's not connected already and
	max\_number in the class is not reached yet.

\item[{compressed}]
	Ziplinks should be used with this server connection.
	This compresses traffic using zlib, saving some bandwidth
	and speeding up netbursts.

	If you have trouble setting up a link, you should
	turn this off as it often hides error messages.

\item[{topicburst}]
	Topics should be bursted to this server.

This is enabled by default.
\end{description}


\subsection{listen \{\} block}

\begin{verbatim}
listen {
	host = "text";
	port = number;
};\end{verbatim}

	A listen block specifies what ports a server should listen on.


{\sc listen \{\} variables}
\nopagebreak

\noindent
\begin{description}
\item[{host}]
	An optional host to bind to. Otherwise, the ircd will listen on all
	available hosts.

\item[{port}]
	A port to listen on. You can specify multiple ports via commas, and
	define a range by seperating the start and end ports with two dots
	(..).

\end{description}

\subsection{modules \{\} block}

\begin{verbatim}
modules {
	path = "text";
	module = text;
};\end{verbatim}

	The modules block specifies information for loadable modules.


{\sc modules \{\} variables}
\nopagebreak

\noindent
\begin{description}
\item[{path}]
	Specifies a path to search for loadable modules.

\item[{module}]
	Specifies a module to load, similar to loadmodule.

\end{description}


\subsection{general \{\} block}

\begin{verbatim}
modules {
	values
};\end{verbatim}

	The general block specifies a variety of options, many of which were in
	\nolinkurl{config.h} in older daemons. The options are documented in
	\nolinkurl{reference.conf}.


\subsection{channel \{\} block}

\begin{verbatim}
modules {
	values
};\end{verbatim}

	The channel block specifies a variety of channel-{}related options,
	many of which were in \nolinkurl{config.h} in older daemons. The
	options are documented in \nolinkurl{reference.conf}.


\subsection{serverhide \{\} block}

\begin{verbatim}
modules {
	values
};\end{verbatim}

	The serverhide block specifies options related to server hiding. The
	options are documented in \nolinkurl{reference.conf}.

\subsection{blacklist \{\} block}

\begin{verbatim}
blacklist {
	host = "text";
	reject_reason = "text";
};\end{verbatim}

	The blacklist block specifies DNS blacklists to check. 	Listed clients
	will not be allowed to connect. IPv6 clients are not checked against
	these.

	Multiple blacklists can be specified, in pairs with first host then
	reject\_reason.

{\sc blacklist \{\} variables}
\nopagebreak

\noindent
\begin{description}
\item[{host}]
	The DNSBL to use.

\item[{reject\_reason}]
	The reason to send to listed clients when disconnecting them.

\end{description}


\subsection{alias \{\} block}

\begin{verbatim}
alias "name" {
	target = "text";
};\end{verbatim}

	Alias blocks allow the definition of custom commands.
	These commands send PRIVMSG to the given target. A real
	command takes precedence above an alias.


{\sc alias \{\} variables}
\nopagebreak

\noindent
\begin{description}
\item[{target}]
	The target nick (must be a network service (umode +S)) or
	user@server.
	In the latter case, the server cannot be this server,
	only opers can use user starting with "opers" reliably and
	the user is interpreted on the target server only
	so you may need to use nick@server instead).
\end{description}


\subsection{cluster \{\} block}

\begin{verbatim}
cluster {
	name = "text";
	flags = list;
};\end{verbatim}

	The cluster block specifies servers we propagate things to
	automatically.
	This does not allow them to set bans, you need a separate shared\{\}
	block for that.

	Having overlapping cluster\{\} items will cause the command to
	be executed twice on the target servers. This is particularly
	undesirable for ban removals.

	The letters in parentheses denote the flags in /stats U.


{\sc cluster \{\} variables}
\nopagebreak

\noindent
\begin{description}
\item[{name}]
	The server name to share with, this may contain wildcards
	and may be stacked.

\item[{flags}]
	The list of what to share, all the name lines above this
	(up to another flags entry) will receive these flags.
	They are listed below.
\end{description}


{\sc cluster \{\} flags}
\nopagebreak

\noindent
\begin{description}
\item[{kline (K)}]
	Permanent K:lines

\item[{tkline (k)}]
	Temporary K:lines

\item[{unkline (U)}]
	K:line removals

\item[{xline (X)}]
	Permanent X:lines

\item[{txline (x)}]
	Temporary X:lines

\item[{unxline (Y)}]
	X:line removals

\item[{resv (Q)}]
	Permanently reserved nicks/channels

\item[{tresv (q)}]
	Temporarily reserved nicks/channels
\item[{unresv (R)}]
	RESV removals

\item[{locops (L)}]
	LOCOPS messages (sharing this with * makes LOCOPS rather
	similar to OPERWALL which is not useful)

\item[{all}] All of the above
\end{description}


\subsection{shared \{\} block}

\begin{verbatim}
shared {
	oper = "user@host", "server";
	flags = list;
};\end{verbatim}

	The shared block specifies opers allowed to perform certain actions
	on our server remotely.
	These are ordered top down. The first one matching will determine
	the oper's access.
	If access is denied, the command will be silently ignored.

	The letters in parentheses denote the flags in /stats U.


{\sc shared \{\} variables}
\nopagebreak

\noindent
\begin{description}
\item[{oper}]
	The user@host the oper must have, and the server they must be on. This
	may contain wildcards.

\item[{flags}]
	The list of what to allow, all the oper lines above this (up to another
	flags entry) will receive these flags. They are listed below.

\end{description}
\notebox{Note}{

	While they have the same names, the flags have subtly different
	meanings from those in the cluster\{\} block.

}

{\sc shared \{\} flags}
\nopagebreak

\noindent
\begin{description}
\item[{kline (K)}]
	Permanent and temporary K:lines

\item[{tkline (k)}]
	Temporary K:lines

\item[{unkline (U)}]
	K:line removals

\item[{xline (X)}]
	Permanent and temporary X:lines

\item[{txline (x)}]
	Temporary X:lines

\item[{unxline (Y)}]
	X:line removals

\item[{resv (Q)}]
	Permanently and temporarily reserved nicks/channels

\item[{tresv (q)}]
	Temporarily reserved nicks/channels

\item[{unresv (R)}]
	RESV removals

\item[{all}]
	All of the above; this does not include locops, rehash, dline, tdline or undline.

\item[{locops (L)}]
	LOCOPS messages (accepting this from * makes LOCOPS rather
	similar to OPERWALL which is not useful); unlike the other flags,
	this can only be accepted from *@* although it can be
	restricted based on source server.

\item[{rehash (H)}]
	REHASH commands; all options can be used

\item[{dline (D)}]
	Permanent and temporary D:lines

\item[{tdline (d)}]
	Temporary D:lines

\item[{undline (E)}]
	D:line removals

\item[{none}]
	Allow nothing to be done
\end{description}

\subsection{service \{\} block}

\begin{verbatim}
service {
	name = "text";
};\end{verbatim}

	The service block specifies privileged servers (services). These
	servers have extra privileges such as setting login names on users
	and introducing clients with umode +S (unkickable, hide channels, etc).
	This does not allow them to set bans, you need a separate shared\{\}
	block for that.

	Do not place normal servers here.


	Multiple names may be specified but there may be only one service\{\}
	block.


{\sc service \{\} variables}
\nopagebreak

\noindent
\begin{description}
\item[{name}]
	The server name to grant special privileges. This may not
	contain wildcards.
\end{description}

\section{Hostname resolution (DNS)}

	SporksIRCd uses solely DNS for all hostname/address lookups
	(no \nolinkurl{/etc/hosts} or anything else).
	The DNS servers are taken from \nolinkurl{/etc/resolv.conf}.
	If this file does not exist or no valid IP addresses are listed in it,
	the local host (127.0.0.1) is used.


	IPv4 as well as IPv6 DNS servers are supported, but it is not
	possible to use both IPv4 and IPv6 in \nolinkurl{/etc/resolv.conf}.


	For both security and performance reasons, it is recommended
	that a caching nameserver such as BIND be run on the same machine
	as SporksIRCd and that \nolinkurl{/etc/resolv.conf} only
	list 127.0.0.1.
