\chapter{Umodes}
\label{umodes}

\section{Meanings of user modes}
\label{umodelist}

\subsection{+a, server administrator}

	This vanity usermode is used to denote a server administrator in WHOIS
	output. All local `admin' privileges are independent of it, though
	services packages may grant extra privileges to +a users.


\subsection{+D, deaf}

\notebox{Note}{

	This is a user umode, which anybody can set. It is not specific to
	operators.
 
}
	Users with the +D umode set will not receive messages sent to channels.
	Joins, parts, topic changes, mode changes, etc are received as normal,
	as are private messages.
 

	Support of this umode is indicated by the DEAF token in
	RPL\_ISUPPORT (005); the parameter indicates the letter
	of the umode. Note that several common IRCD implementations have
	an umode like this (typically +d) but do not have the token in 005.


\subsection{+g, Caller ID}

\notebox{Note}{

	This is a user umode, which anybody can set. It is not specific to
	operators.

}
	Users with the +g umode set will only receive private messages from
	users on a session-{}defined whitelist, defined by the /accept command.
	If a user who is not on the whitelist attempts to send a private
	message, the target user will receive a rate-{}limited notice saying
	that the user wishes to speak to them.
 

	If a network operator needs to speak to a user who has +g set, the
	operator can use the OACCEPT command to be able to speak to the user.
 

	Support of this umode is indicated by the CALLERID token in
	RPL\_ISUPPORT (005); the optional parameter indicates the letter of the
	umode, otherwise +g.


\subsection{+i, invisible}

\notebox{Note}{

	This is a user umode, which anybody can set. It is not specific to operators.
 
}
	Invisible users do not show up in WHO and NAMES unless you can see them.
 

\subsection{+l, receive locops}

	LOCOPS is a version of OPERWALL that is sent to opers on a single
	server only. With cluster\{\} and shared\{\} blocks they can optionally
	be propagated further.
 

	Unlike OPERWALL, any oper can send and receive LOCOPS.


\subsection{+o, operator}

	This indicates global operator status.


\subsection{+p, override}

	This gives an operator implicit channel operator privileges in all
	channels. Note that +p expires after a while; set the
	expire\_override\_time directive in ircd.conf to determine how long.


\subsection{+Q, disable forwarding}

\notebox{Note}{

	This is a user umode, which anybody can set. It is not specific to
	operators.
 
}
	This umode prevents you from being affected by channel forwarding. If
	enabled on a channel, channel forwarding sends you to another channel
	if you could not join. See channel mode +f for more information.


\subsection{+R, reject messages from unauthenticated users}

\notebox{Note}{

	This is a user umode, which anybody can set. It is not specific to
	operators.
 
}
	If a user has the +R umode set, then any users who are not
	authenticated will receive an error message if they attempt to send a
	private message or notice to the +R user.
 

	Opers and accepted users (like in +g) are exempt. Unlike +g, the target
	user is not notified of failed messages.


\subsection{+s, receive server notices}

	This umode allows an oper to receive server notices. The requested
	types of server notices are specified as a parameter (`snomask') to
	this umode.


\subsection{+S, network service}

\notebox{Note}{

	This umode can only be set by servers named in a service\{\}
	lock.
 
}
	This umode grants various features useful for services. For example,
	clients with this umode cannot be kicked or deopped on channels, can
	send to any channel, do not show channels in WHOIS, can be the target
	of services aliases and do not appear in /stats p. No server notices
	are sent for hostname changes by services clients; server notices about
	kills are sent to snomask +k instead of +s.

	The exact effects of this umode are variable; no user or oper on an
	actual SporksIRCD server can set it.
 

\subsection{+w, receive wallops}

\notebox{Note}{

	This is a user umode, which anybody can set. It is not specific to operators.
 
}
	Users with the +w umode set will receive WALLOPS messages sent by
	opers. Opers with +w additionally receive WALLOPS sent by servers (e.g.
	remote CONNECT, remote SQUIT, various severe misconfigurations,	many
	services packages).


\subsection{+z, receive operwall}

	OPERWALL differs from WALLOPS in that the ability to receive such
	messages is restricted. Opers with +z set will receive OPERWALL
	messages.


\subsection{+Z, SSL user}

	This umode is set on clients connected via SSL/TLS. It cannot be set or
	unset after initial connection.


\section{Snomask usage}
\label{snomaskusage}

	Usage is as follows:

\command{MODE}
    \userliteral{nick} \literal{+s} \literal{+\slash-{}}\userliteral{flags}

	To set snomasks.

\command{MODE}
    \userliteral{nick} \literal{-{}s}

	To clear all snomasks.

	Umode +s will be set if at least one snomask is set.

	Umode +s is oper only by default, but even if you allow nonopers to set
	it, they will not get any server notices.


\section{Meanings of server notice masks}
\label{snomasklist}

\subsection{+b, bot warnings}

	Opers with the +b snomask set will receive warning messages from the
	server when potential flooders and spambots are detected.


\subsection{+c, client connections}

	Opers who have the +c snomask set will receive server notices when
	clients attach to the local server.


\subsection{+C, extended client connection notices}

	Opers who have the +C snomask set will receive server notices when
	clients attach to the local server. Unlike the +c snomask, the
	information is displayed in a format intended to be parsed by scripts,
	and includes the two unused fields of the USER command.


\subsection{+d, debug}

	The +d snomask provides opers extra information which may be of
	interest to debuggers. It will also cause the user to receive server
	notices if certain assertions fail inside the server. Its precise
	meaning is variable. Do not depend on the effects of this snomask as
	they can and will change without notice in later revisions.


\subsection{+f, full warning}

	Opers with the +f snomask set will receive notices when a user
	connection is denied because a connection limit is exceeded (one of the
	limits in a class\{\} block, or the total per-{}server limit settable
	with /quote set max).


\subsection{+F, far client connection notices}

\notebox{Note}{

	This snomask is only available if the \nolinkurl{sno_farconnect.so}
	extension is loaded.
 
}
	Opers with +F receive server notices when clients connect or disconnect
	on other servers. The notices have the same format as those from the +c
	snomask, except that the class is ? and the source server of the notice
	is the server the user is/was on.

	No notices are generated for netsplits and netjoins. Hence, these
	notices cannot be used to keep track of all clients on the
	network.

	There is no far equivalent of the +C snomask.


\subsection{+k, server kill notices}

	Opers with the +k snomask set will receive server notices when services
	kill users and when other servers kill and SAVE (forced nick change to
	UID) users. Kills and saves by this server are on +d or +s.


\subsection{+n, nick change notices}

	An oper with +n set will receive a server notice every time a local
	user changes their nick, giving the old and new nicks. This is mostly
	useful for bots that track all users on a single server.


\subsection{+r, notices on name rejections}

	Opers with this snomask set will receive a server notice when somebody
	tries to use an invalid username, or if a dumb HTTP proxy tries to
	connect.


\subsection{+s, generic server notices}

	This snomask allows an oper to receive generic server notices. This
	includes kills from opers (except services).


\subsection{+u, unauthorized connections}

	This snomask allows an oper to see when users try to connect who do not
	have an available auth\{\} block.


\subsection{+W, whois notifications}

\notebox{Note}{

	This snomask is only available if the \nolinkurl{sno_whois.so}
	extension is loaded.

}
	Opers with +W receive notices when a WHOIS is executed on them on
	their server (showing idle time).
 

\subsection{+x, extra routing notices}

	Opers who have the +x snomask set will get notices about servers
	connecting and disconnecting on the whole network. This includes all
	servers connected behind the affected link. This can get rather noisy
	but is useful for keeping track of all linked servers.


\subsection{+y, spy}

	Opers with +y receive notices when users try to join RESV'ed (`juped')
	channels. Additionally, if certain extension modules are loaded, they
	will receive notices when special commands are used.


\subsection{+Z, operspy notices}

	Opers with +Z receive notices whenever an oper anywhere on the network
	uses operspy.
 
	This snomask can be configured to be only effective for admins.
