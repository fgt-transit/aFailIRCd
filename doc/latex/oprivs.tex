\chapter{Oper privileges}
\label{oprivs}

\section{Meanings of oper privileges}
\label{oprivlist}

These are specified in privset\{\}.


\subsection{oper:admin, server administrator}

	Various privileges intended for server administrators. Among other
	things, this automatically sets umode +a and allows loading modules.


\subsection{oper:die, die and restart}

	This grants permission to use DIE and RESTART, shutting down
	or restarting the server.


\subsection{oper:global\_force, global force}

	Allows using FORCE* commands on users on any server.


\subsection{oper:global\_kill, global kill}

	Allows using KILL on users on any server.


\subsection{oper:hidden, hide from /stats p}

	This privilege currently does nothing, but was designed	to hide bots
	from /stats p so users will not message them for help.


\subsection{oper:hidden\_admin, hidden administrator}

	This grants everything granted to the oper:admin privilege, except the
	ability to set umode +a. If both oper:admin and	oper:hidden\_admin are
	possessed, umode +a can still not be used.


\subsection{oper:kline, kline and dline}

	Allows using KLINE and DLINE, to ban users by user@host mask
	or IP address.


\subsection{oper:local\_force, local force}

This allows the use of FORCE* commands on users on the same server.


\subsection{oper:local\_kill, kill local users}

	This grants permission to use KILL on users on the same server,
	disconnecting them from the network.


\subsection{oper:mass\_notice, global notices and wallops}

	Allows using server name (\$\$mask) and hostname (\$\#mask) masks in
	NOTICE and PRIVMSG to send a message to all matching users, and allows
	using the WALLOPS command to send a message to all users with umode +w
	set.


\subsection{oper:operwall, send/receive operwall}

	Allows using the OPERWALL command and umode +z to send and receive
	operwalls.


\subsection{oper:rehash, rehash}

	Allows using the REHASH command, to rehash various configuration files
	or clear certain lists.


\subsection{oper:remoteban, set remote bans}

	This grants the ability to use the ON argument on
	DLINE/KLINE/XLINE/RESV and UNDLINE/UNKLINE/UNXLINE/UNRESV to set and
	unset bans on other servers, and the server argument on REHASH. This
	is only allowed if the oper may perform the action locally, and if the
	remote server has a shared\{\} block.

\notebox{Note}{

	If a cluster\{\} block is present, bans are sent remotely even if the
	oper does not have oper:remoteban privilege.

}

\subsection{oper:resv, channel control}

	This allows using /resv, /unresv and changing the channel modes +L and
	+P.


\subsection{oper:routing, remote routing}

	This allows using the third argument of the CONNECT command, to
	instruct another server to connect somewhere, and using SQUIT with an
	argument that is not locally connected. (In both cases all opers with
	+w set will be notified.)


\subsection{oper:override, operator access in all channels}

	This allows the operator to have implicit operator access in all
	in all channels. It is enabled by setting user mode +p which
	will automatically expire after a period of time set in the
	configuration file.


\subsection{oper:spy, use operspy}

	This allows using /mode !\#channel, /whois !nick, /who !\#channel,
	/chantrace !\#channel, /topic !\#channel, /who !mask, /masktrace
	!user@host :gecos and /scan umodes +modes-{}modes global list to see
	through secret channels, invisible users, etc.

	All operspy usage is broadcasted to opers with snomask +Z set (on the
	entire network) and optionally logged. If you grant this to anyone, it
	is a good idea to establish concrete policies describing what it is to
	be used for, and what not.

	If operspy\_dont\_care\_user\_info is enabled, /who mask is operspy
	also, and /who !mask, /who mask, /masktrace !user@host :gecos and
	/scan umodes +modes-{}modes global list do not generate +Z notices or
	logs.


\subsection{oper:unkline, unkline and undline}

	Allows using UNKLINE and UNDLINE.


\subsection{oper:xline, xline and unxline}

	Allows using XLINE and UNXLINE, to ban/unban users by realname.


\subsection{snomask:nick\_changes, see nick changes}

	Allows using snomask +n to see local client nick changes. This is
	designed for monitor bots.
