\chapter{User Commands}
\label{ucommands}

\section{User commands}

	Standard IRC commands are not listed here. Several of the commands in
	the operator commands chapter can also be used by normal users.


\subsection{ACCEPT}

\command{ACCEPT}
    \userliteral{nick}\literal{,-{}}\userliteral{nick}\literal{,}
    \userliteral{...}

	Adds or removes users from your accept list for umode +g and +R. Users
	are automatically removed when they quit, split or change nick.


\command{ACCEPT}
    \literal{*}

	Lists all users on your accept list.

	Support of this command is indicated by the CALLERID token in
	RPL\_ISUPPORT (005); the optional parameter indicates the letter of the
	`only allow accept users to send private messages' umode, otherwise +g.
	In SporksIRCD this is always +g.


\subsection{CNOTICE}

\command{CNOTICE}
    \userliteral{nick} \userliteral{channel} \literal{:}\userliteral{text}

	Providing you are opped (+o) or voiced (+v) in \userliteral{channel},
	and \userliteral{nick} is a member of \userliteral{channel}, CNOTICE
	generates a NOTICE towards \userliteral{nick}.

	CNOTICE bypasses any anti-{}spam measures in place. If you get `Targets
	changing too fast, message dropped', you should probably use this
	command, for example sending a notice to every user joining a certain
	channel.

	Note: NOTICE automatically behaves as CNOTICE if you are in a channel
	fulfilling the conditions.

	Support of this command is indicated by the CNOTICE token in
	RPL\_ISUPPORT (005).


\subsection{CPRIVMSG}

\command{CPRIVMSG}
    \userliteral{nick} \userliteral{channel} \literal{:}\userliteral{text}

	Providing you are opped (+o) or voiced (+v) in \userliteral{channel},
	and \userliteral{nick} is a member of \userliteral{channel}, CPRIVMSG
	generates a PRIVMSG towards \userliteral{nick}.

	CPRIVMSG bypasses any anti-{}spam measures in place. If you get
	`Targets changing too fast, message dropped', you should probably use
	this command.

	Note: PRIVMSG automatically behaves as CPRIVMSG	if you are in a channel
	fulfilling the conditions.

	Support of this command is indicated by the CPRIVMSG token in
	RPL\_ISUPPORT (005).


\subsection{ROLEPLAY}

	ROLEPLAY consists of a number of similar commands, all of which send
	messages to channel under 'fake' nicknames. The user of the command can
	be identified by looking at the ident of the 'fake' user - it will
	always be the nickname of the person using the command, and the host
	will always be npc.fakeuser.invalid. These commands require the target
	channel to be set +N, and obey the same restrictions as regular
	PRIVMSG.

\notebox{Note}{
	These commands are only enabled with m\_roleplay.so loaded.
}

\subsubsection{NPC}
\command {NPC}
    \userliteral{channel} \userliteral{nick} \literal{:} \userliteral{message}

	Sends a standard roleplay message to \userliteral{channel} with the
	\userliteral{nick} specified specified.

\subsubsection{NPCA}
\command{NPCA}
    \userliteral{channel} \userliteral{nick} \literal{:} \userliteral{message}

	Sends a CTCP action message to \userliteral{channel} with the 
	\userliteral{nick} specified.

\subsubsection{SCENE}
\command{SCENE}
    \userliteral{channel} \literal{:} \userliteral{message}

	The same as NPC, except the message always originates from the
	nickname =Scene=, which will not be underlined like the other commands.

\subsubsection{FSAY}
\command{FSAY}
    \userliteral{channel} \userliteral{nick} \literal{:} \userliteral{message}

	This command is exactly like NPC, except the nickname will not be
	underlined if the person running the command is an operator.

\notebox{Note}{
	This command is disabled by default unless -DABUSIVE\_ROLEPLAY is in
	your CFLAGS.
}

\subsubsection{FACTION}
\command{FACTION}
    \userliteral{channel} \userliteral{nick} \literal{:} \userliteral{message}

	This command is exactly like NPCA, except the nickname will not be
	underlined if the person running the command is an operator.

\notebox{Note}{
	This command is disabled by default unless -DABUSIVE\_ROLEPLAY is in
	your CFLAGS.
}


\subsection{FINDFORWARDS}

\command{FINDFORWARDS}
    \userliteral{channel}

\notebox{Note}{

	This command is only available if the \nolinkurl{m_findforwards.so}
	extension is loaded.

}
	Displays which channels forward to the given channel (via cmode +f). If
	there are very many channels the list will be truncated.

	You must be a channel operator on the channel or an IRC operator to use
	this command.


\subsection{HELP}

\command{HELP}
    [\userliteral{topic}]

	Displays help information. \userliteral{topic} can be \literal{INDEX},
	\literal{CREDITS}, \literal{UMODE}, \literal{CMODE}, \literal{SNOMASK},
	or a command name.


	There are separate help files for users and opers. Opers can use UHELP
	to query the user help files.


\subsection{IDENTIFY}

\command{IDENTIFY}
    \userliteral{parameters...}

\notebox{Note}{

	This command is only available if the \nolinkurl{m_identify.so}
	extension is loaded.

}
	Sends an identify command to either NickServ or ChanServ. If the first
	parameter starts with \#, the command is sent to ChanServ, otherwise to
	NickServ. The word IDENTIFY, a space and all parameters are
	concatenated and sent as a PRIVMSG to the service. If the service is
	not online or does not have umode +S set, no message will be sent.

	The exact syntax for this command depends on the services package in
	use.


\subsection{KNOCK}

\command{KNOCK}
    \userliteral{channel}

	Requests an invite to the given channel. The channel must be locked
	somehow (+ikl), must not be +p and you may not be banned or quieted.
	Also, this command is rate limited.

	If successful, all channel operators will receive a 710 numeric. The
	recipient field of this numeric is the channel.

	Support of this command is indicated by the KNOCK token in
	RPL\_ISUPPORT (005).


\subsection{MONITOR}

	Server side notify list. This list contains nicks. When a user
	connects, quits with a listed nick or changes to or from a listed nick,
	you will receive a 730 numeric if the nick went online and a 731
	numeric if the nick went offline.

	Support of this command is indicated by the MONITOR token in
	RPL\_ISUPPORT (005); the parameter indicates the maximum number of
	nicknames you may have in your monitor list.

	You may only use this command once per second.

	More details can be found in \nolinkurl{doc/monitor.txt} in the source
	distribution.


\command{MONITOR +}
    \userliteral{nick}\literal{,}\userliteral{...}

	Adds nicks to your monitor list. You will receive 730 and 731 numerics
	for the nicks.


\command{MONITOR -{}}
    \userliteral{nick}\literal{,}\userliteral{...}

	Removes nicks from your monitor list. No output is generated for this
	command.


\command{MONITOR C}

	Clears your monitor list. No output is generated for this command.


\command{MONITOR L}

	Lists all nicks on your monitor list, using 732 numerics and ending
	with a 733 numeric.


\command{MONITOR S}

	Shows status for all nicks on your monitor list, using 730 and 731
	numerics.
